My master thesis is about investigating n-type dopant diffusion in the semiconductor $\beta$-Ga$_2$O$_3$. $\beta$-Ga$_2$O$_3$ is a promising material in 'power electronics' \cite{} and, as with all semiconductors, he doping is important to understand. The n-type dopant diffusion is believed to be oxygen vacancy aided and dependent. That makes the oxygen vacancies interesting to study. In this project the three different oxygen vacancies in $\beta$-Ga$_2$O$_3$ were studied and compared with density functional theory.

We started with convergence tests of the primitive unit $\beta$-Ga$_2$O$_3$, both with respect to cut-off energy and k-point density. We also looked at the density of states and plotted the band structure of $\beta$-Ga$_2$O$_3$.

After that, we increased the unit cell size to a super cell. This, to be able to insert an oxygen vacancy. We relaxed the structure and calculated the energy of the bulk $\beta$-Ga$_2$O$_3$. We made three different supercells each with an oxygen vacancy at different oxygen sites. We relaxed the structure and then calculated the total energy. To find the formation energy of the different oxygen vacancies, the energy of an oxygen molecule in vacuum was calculated as well. 

At last, after finding the oxygen vacancy with minimum formation energy, local density of state and electron density isosurfaces were used to investigate the oxygen vacancies further. 
